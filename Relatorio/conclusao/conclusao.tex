Fazendo uma apreciação geral do resultado final, podemos concluir que, como todo o \textit{input} necessário se encontra devidamente anotado, a modelação e análise do sistema (domínio, \textit{use cases} e estado) facilita imenso a criação da interface da aplicação e, posteriormente, a escrita do código \textit{backend}. Isto permitirá desenvolver o software de uma forma mais eficaz, devido à redução da probabilidade de aparecimento de erros devido a mau planeamento, o que, em última análise, facilitou o desenvolvimento e diminui bastante o tempo necessário a este.