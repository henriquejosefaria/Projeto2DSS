O objetivo deste trabalho é analisar e modelar uma aplicação \textit{desktop} de gestão e comunicação entre um stand e uma fábrica e, posteriormente, codificar e executar esta mesma aplicação. Para isto, o software deve:
\begin{itemize}
    \item permitir que um funcionário de stand, junto do cliente, consiga configurar e enviar uma encomenda de um novo carro para a fábrica;
    \item ter pacotes de configuração predefinidos e com desconto;
    \item propor a melhor configuração possível perante um limite máximo a gastar;
    \item verificar se um componente obriga à colocação de mais componentes; 
    \item permitir guardar e retomar uma configuração pendente;
    \item indicar os carros a serem produzidos por ordem de chegada de configurações;
    \item permitir ao funcionário de fábrica atualizar o stock de componentes.
\end{itemize}

\noindent
Além disto, o software \underline{não} deve:

\begin{itemize}
    \item permitir que componentes incompatíveis sejam incluídos na mesma encomenda;
    \item permitir que uma encomenda seja enviada sem possuir todos os componentes obrigatórios;
    \item permitir que uma encomenda seja enviada quando não existe um número de componentes suficiente em stock para a satisfazer;
\end{itemize}

\noindent
Neste relatório, ilustraremos o processo de que usamos para planear e, posteriormente, programar a aplicação, tomando sempre em consideração os requisitos enunciados acima.